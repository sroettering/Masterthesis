%************************************************
\chapter{Related Work}\label{ch:relatedwork}
%************************************************

\begin{itemize}
\item{
	Casual Interaction Henning Pohl, focused-casual continuum
}
\item{
	Karo Busse, Light control bracelet,
	silicone cast, teensy microcontroller, accelerometer, bluetooth, capacitive touch panel,
	material and design is important, users should be able to wear the device a lot
}
\item{
	Vatavu User-Defined Gestures for Free-Hand TV Control,
	Agreement Analysis on 12 user defined tv control gestures,
	simple free-hand gestures found
}
\item{
	ShoeSense, Depth-sensor mounted on shoe pointing upwards,
	discreet and demonstrative hand gestures possible,
	investigated social acceptability, physical and mental demand, and user preference
}
\item{
	Gestural and Audio Metaphors as a Means of Control,
	mobile music player controllable by touch-gestures and non speech audio
}
\item{
	Blumendorf, Multimodal smart home user interfaces,
	interaction commands must be learnable easily and quickly, user prefers short commands over complex sentences in speech interaction
}
\item{
	I'm Home: Smartphone-enabled gestural interaction with multi-modal smart-home systems,
	accelerometer based gesture control of tv hifi-system, lamp and window-shutter
}
\item{
	A Gesture Based System for Context - Sensitive Interaction with Smart Homes,
	3D acceleration gestures, wiimote, gestures have different meanings in different contexts, reducing gesture set size
}
\item{
	Can user-derived gesture be considered as the best,
	30 participants, created own gesture set for controlling tv and air conditioning, in second test every user was presented all other gestures, highest agreed gesture from first test was not highest in second, since most users did not come up with some creative gestures.
}
\item{
	Engaging with Mobile Music Retrieval,
	Touch-based user interface for music retrieval
}

\end{itemize}

Casual interaction has become a big research topic in \ac{HCI} nowadays.

Pohl and Murray-Smith \cite{pohl2013focused} have described the \textit{focused-casual continuum}, which is a control-theoretic framework that characterizes input techniques in regard to how much flexibility, in terms of thinking and work, they allow a user to invest into interactions. They showed in a user study that users adjust the level of engagement to the task's complexity. Also Pohl \dots
%TODO

On this basis, \cite{Busse2014Thesis} constructed a wrist worn silicone bracelet. When worn, a user could casually interact with a light source. Simple actions like turning the light on and off up to picking individual colors with a capacitive touch stripe. Accelerometer based gestures could be used to activate previously defined and memorized light settings. Despite being highly accessible on the wrist, a user would still have to utilize the hand without the bracelet to activate it's features making interactions rather impractical in certain situations.

Another approach places a depth camera for capturing hand gestures on the user's foot pointing upwards \cite{bailly2012shoesense}. This allows for discreet interactions thus neglecting concerns of social acceptability of performing gestures as they found out. Mention user study in different scenarios \dots

%Chapter \ref{ch:relatedwork} 


%*****************************************
%*****************************************
%*****************************************
%*****************************************
%*****************************************




