%************************************************
\chapter{Conclusion and future work}\label{ch:conclusion}
%************************************************

This thesis explores casual interaction with a smartwatch. Therefor, a mobile music player is implemented on smartphone as a representative application. Three different input techniques namely touch, speech commands and armgestures covering a wide range of the focused-casual continuum enable the user to control the music player with the smartwatch regardless of the current context. In a user study with ten participants the input techniques and how they were implemented are verified with respect to their convenience. The participants are observed interacting with the music player in six common everyday scenarios and asked to rate different aspects of the interaction techniques afterwards.

Armgestures turn out to be avoided and they receive a poor rating as well which can be reasoned with high error rates due to insufficient training and the decision of adding an activation gesture followed by a vibration countdown. Concentrating on when to perform the gesture draws more attention from the current task than intended.

Touch input on the smartwatch is favored in contexts with mental distraction for simple tasks like play and pause. For complex tasks touch input is used when a distraction from the current task has no significant consequences. Most importantly users choose touch as a reliable fallback in case of other techniques malfunctioning. 

Speech commands are surprisingly popular and high rated for their intuitive operability and their lack of demanding attention. They cover a wide area of the focused-casual continuum with their control possibilities and are usable in many different situations. Speech recognition is, however, vulnerable to loud background noise which sets a limitation to this interaction technique and shows a demand for alternatives to use. Nonetheless, it is a flexible and powerful technique that can allow users to interact with a system by just expressing their demand.

Overall a smartwatch is a suitable device for offering casual interaction regardless of the supported interaction techniques due to being highly accessible and enabling quick interactions. The evaluation shows that hands-free interaction is possible with a smartwatch supporting speech and gestures as input. Needless to say, the casual interaction system chosen for the music player has room for further improvements and research. Gesture input can be enhanced by including wrist or even finger gestures which would introduce more complex interaction possiblities. Moreover, a different activation system for gestures or even continuous recognition should be considered. It is also imagineable to add additional sensor hardware to the smartwatch such as a proximity sensor or a camera for enabling around-device interactions that would require even less attention from the user. Additionally, casual interaction could profit from a context aware system that reacts on user input by combining its intrinsic and environmental state and then inferring a corresponding action.


%Chapter \ref{ch:conclusion} 


%*****************************************
%*****************************************
%*****************************************
%*****************************************
%*****************************************




