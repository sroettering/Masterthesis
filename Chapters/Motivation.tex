%************************************************
\chapter{Motivation}\label{ch:motivation}
%************************************************

Electronic devices connected to the internet play a big role in almost everyone's life. According to a survey\footnote{\url{http://www.pewglobal.org/2016/02/22/smartphone-ownership-and-internet-usage-continues-to-climb-in-emerging-economies/}} two thirds of the global population have access to the internet and almost every third human owns a smartphone in 2016. People not being bound to stationary computers anymore has numerous advantages and opens up for manifold possibilites of applications. However most applications are exclusively touch based and hence draw a user's attention from her environment to the screen in a large extent. Depending on the context, this inobservance c various levels of adverse impact -- one merely could have to reread a passage of a book or in fact cause a serious car accident with lethal consequences. Furthermore, a user is not always able to interact with a handheld device as intended due to physical or mental barriers.
 
\textit{Casual Interaction} addresses this issue by introducing additional input techniques and interactions which offer the required flexibility in terms of control over an application, such that the user can limit the required amount of focus to a minimum \cite{pohl2013focused}. The H-metaphor \cite{flemisch2003h} expresses this by looking at a horse rider. Whenever high amount of control is needed, the rider can steer the horse precisely with the reins. However, riding straightly requires nearly no effort from the rider. 

The music player implemented in this thesis offers a speech and armgesture interface for input in addition to the typical touch based user interface. This enables the user to leave her handheld device in the pocket and interact with it even when her hands are not available. A typical daily life scenario involving the music player might be the following:\\

\begin{addmargin}[1cm]{1cm}
Programmer Bob is a passionate music listener. At work he starts the day with his favorite playlist with shuffle mode enabled to boost his coding efficiency. Occasionally some songs hit his ears that he heard enough of in the last days so he performs a short arm swing to the right to skip to the next song without loosing track of his current programming task. 

After work he rides his bike home, still listening to music, to have a physical compensation for sitting in a chair half the day. In order to hear other traffic participants and emergency sirens he decides to turn the volume of the music down a bit by raising his smartwatch to his mouth and saying ``\textit{volume down, please}''. Arriving at home he switches from listening with headphones to his brand new sound system that is spread around his entire house. During dinner preparations he gets a phone call from a friend. His phone resides one the table, however Bob's hands are still dirty from cooking so he tells his watch to pause the music and answer the call. They arrange on having dinner together today at Bob's place. It strikes Bob that his music library is not prepared for such a dinner but he is able to delegate the creation of a playlist to his music player by simply specifying suitable audio features. Not having to bother about finding the right music enables Bob to finish cooking just before his guest arrives.\\
\end{addmargin}

This fictive scenario reveals the necessity of alternative input techniques to perform casual interactions in situations where users are physically or mentally obstructed. In order to decide which additional input techniques an application should provide and how the corresponding user interfaces should be designed one has to consider the possibilities these techniques offer as well as how users approach them. The last aspect particularly depends on the user's preferences and perception performing the interactions. On the one hand, the perceived amount of control for a particular level of engagement is important. \textit{Is she able to achieve the desired reaction of the application or does it feel like the application has developed it's own life?}

On the other hand, users often times get influenced by how interactions appear to the environment. \textit{Can i perform an arm gesture right now or will people stare at me if i suddenly wave my arm through the air?}
These and related concerns need to be kept in mind in order to be able to develop useful and effective casual user interfaces.

However, devices require certain hardware features to enable such interactions in the first place. First, it should stay where it is needed without encumbering the user. Typical remotes or smartphones occupy at least one hand for every interaction they offer. Since this is not beneficial a wearable device is needed that is attached to the body without obstructing everyday activities.
Second, the device should offer touch-free interaction. This can be realised by adding movement sensors (e.g. accelerometer or gyroscope) and a microphone.\cite{Busse2014Thesis}

This thesis builds on the previous work of Karoline Busse \cite{Busse2014Thesis} who developed a wrist-worn silicone bracelet intended for usage with lighting systems. The bracelet is missing on some important components, though, to gain more potential, namely a microphone and a display. Smartwatches basically offer the most important hardware components needed for creating a comfortable and enjoyable casual interaction experience thus being a perfect device for the further studies of this thesis.

A music player is chosen as a representative everyday application. The music player is connected to a private Spotify account via the Spotify Android \ac{SDK} \footnote{\url{https://developer.spotify.com/technologies/spotify-android-sdk/}} which serves the music library. Touch, speech and arm gesture input for player control realize different levels of engagement.

Chapter \ref{ch:relatedwork} first outlines the related work. Chapter \ref{ch:implementation} then gives an insight into the implementation of the music player's core features. Subsequently the user study design is addressed in chapter \ref{ch:user-study} and the resultant data is evaluated in chapter \ref{ch:evaluation}.

Finally, chapter \ref{ch:conclusion} discusses the findings, draws a conclusion and provides ideas for future improvements to casual interactions.

%Chapter \ref{ch:motivation} 


%*****************************************
%*****************************************
%*****************************************
%*****************************************
%*****************************************




