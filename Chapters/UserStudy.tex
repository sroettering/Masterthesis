%************************************************
\chapter{User Study}\label{ch:userStudy}
%************************************************

% Intro text, explain why user study, explain experiment form, etc

% Introduce scenarios and tasks. maybe both in tables. 

% include experiment assets such as route images or book chapter.

% mention questionnaire

% data evaluation

% meta values, like age and gender.

% plot priority data

% plot 0-10 scale data

The purpose of the music player described in chapter \ref{ch:implementation} is to examine user behaviour regarding the focused-casual continuum. The idea behind this concept is that users can vary their level of engagement for a certain action while not being bound to interact strictly focused or casual with the device. The three interaction techniques introduced in section \ref{sec:UserInterface} and the variable amount of control they offer to the user create different levels in the focused-casual continuum. 

However, the viability and beneficing of the system needs to be confirmed by users in form of a study. Ten participants, eight male and two female, with ages from 21 to 29 (avg. mean.) were invited to use the music player in six different scenarios. In each scenario the participants were sometimes more, sometimes less either physically or mentally distracted. To force interaction with the music player, the participants were tasked with the same eight coarsely formulated player interactions in each scenario.

%Chapter \ref{ch:relatedwork} 


%*****************************************
%*****************************************
%*****************************************
%*****************************************
%*****************************************




