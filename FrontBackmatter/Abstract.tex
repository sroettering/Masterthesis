%*******************************************************
% Abstract
%*******************************************************
%\renewcommand{\abstractname}{Abstract}
\pdfbookmark[1]{Abstract}{Abstract}
\begingroup
\let\clearpage\relax
\let\cleardoublepage\relax
\let\cleardoublepage\relax

\chapter*{Abstract}
%This thesis deals with \textit{Casual Interaction} with \textit{Smart Home} appliances used in everyday life.
This thesis investigates which and how user interfaces allowing for \textit{casual interaction} should be implemented. A mobile music player as a representative application is implemented on an android handheld device, which is fully controllable by an android smartwatch. Interactions between user and music player can happen via touch input, speech commands or arm gestures. Subsequently a case study investigates which interaction techniques are preferred by users depending on task and context. For this purpose the participants are put in 6 different scenarios in which they have to follow predefined actions regarding the music player. The results show, that speech commands are highly versatile and the use of touch or gesture input depends on the current context of the user.


\vfill

\pdfbookmark[1]{Zusammenfassung}{Zusammenfassung}
\chapter*{Zusammenfassung}
%Diese Arbeit besch\"aftigt sich mit \textit{Casual Interaction} mit \textit{Smart Home} Anwendungen im allt\"aglichen Leben.
Diese Arbeit untersucht welche und auf welche Weise Benutzerschnitt-stellen, die \textit{casual interaction} unterst\"utzen, implementiert werden sollten. Als Beispielanwendung wird ein mobiler Musikplayer auf einem Android Smartphone implementiert, der vollst\"andig per Smartwatch gesteuert werden kann. Touch-Eingabe, Sprachbefehle und Armgesten stehen dem Benutzer dabei als Interaktionsm\"oglichkeiten zur Verf\"ugung. In einer Studie wird anschlie\ss{}end untersucht, welche Interaktionsm\"oglichkeiten, abh\"angig von Aufgabe und Kontext, von den Benutzern bevorzugt werden. Dazu werden die Teilnehmer in 6 verschiedene allt\"agliche Situationen versetzt, in denen sie Vorgaben erhalten, welche Funktionen des Musikplayer sie steuern sollen. Die Resultate zeigen, dass die Sprachsteuerung vielseitig einsetzbar ist und der Einsatz von Touch- bzw. Gesteneingabe vom aktuellen Kontext des Benutzers abh\"angt.


\endgroup			

\vfill