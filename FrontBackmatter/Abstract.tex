%*******************************************************
% Abstract
%*******************************************************
%\renewcommand{\abstractname}{Abstract}
\pdfbookmark[1]{Abstract}{Abstract}
\begingroup
\let\clearpage\relax
\let\cleardoublepage\relax
\let\cleardoublepage\relax

\chapter*{Abstract}
This thesis deals with \textit{Casual Interaction} in \textit{Smart Home} appliances used in everyday life. A mobile music player as an appliance representative is implemented on an android handheld device, which is fully controllable by an android smartwatch. Interactions between user and music player can happen via touch input, speech commands or arm gestures. Subsequently a case study investigates which interaction techniques are preferred by users depending on task and context. For this purpose the participants are put in 6 different scenarios in which they have to follow predefined actions regarding the music player.


\vfill

\pdfbookmark[1]{Zusammenfassung}{Zusammenfassung}
\chapter*{Zusammenfassung}
Diese Arbeit beschäftigt sich mit \textit{Casual Interaction} in \textit{Smart Home} Anwendungen im alltäglichen Leben. Als Beispielanwendung wird ein mobiler Musikplayer auf einem Android Smartphone implementiert, der vollständig per Android Smartwatch gesteuert werden kann. Touch-Eingabe, Sprachbefehle und Armgesten stehen dem Benutzer dabei als Interaktionsmöglichkeiten zur Verfügung. In einer Studie wird anschließend untersucht, welche Interaktionsmöglichkeiten, abhängig von Aufgabe und Kontext, von den Benutzern bevorzugt werden. Dazu werden die Teilnehmer in 6 verschiedene alltägliche Situationen versetzt, in denen sie Vorgaben erhalten, welche Funktionen des Musikplayer sie steuern sollen.


\endgroup			

\vfill